\documentclass[12pt,a4paper]{article}

% ===========================
% 基本设置
% ===========================
\usepackage[utf8]{inputenc}
\usepackage{ctex} % 中文支持,如果只写英文可去掉
\usepackage{amsmath, amssymb, amsthm} % 数学公式
\usepackage{graphicx} % 插图
\usepackage{hyperref} % 超链接
\usepackage{listings} % 代码高亮
\usepackage{xcolor} % 颜色
\usepackage{geometry} % 页边距
\geometry{left=2.5cm,right=2.5cm,top=2.5cm,bottom=2.5cm}

% ===========================
% 代码环境设置
% ===========================
\definecolor{codegray}{rgb}{0.5,0.5,0.5}
\definecolor{codepurple}{rgb}{0.58,0,0.82}
\definecolor{backcolour}{rgb}{0.95,0.95,0.92}

\lstdefinestyle{mystyle}{
    backgroundcolor=\color{backcolour},   
    commentstyle=\color{green!50!black},
    keywordstyle=\color{blue},
    numberstyle=\tiny\color{codegray},
    stringstyle=\color{codepurple},
    basicstyle=\ttfamily\footnotesize,
    breaklines=true,                 
    captionpos=b,                    
    numbers=left,                    
    numbersep=5pt,                  
    showspaces=false,                
    showstringspaces=false,
    showtabs=false,
    tabsize=4
}
\lstset{style=mystyle}

% ===========================
% 文档信息
% ===========================
\title{学习笔记整理 \\ \large(机器学习 / 强化学习 / 大模型)}
\author{Your Name}
\date{\today}

% ===========================
% 正文开始
% ===========================
\begin{document}

\maketitle
\tableofcontents
\newpage

% ---------------------------
\section{课程概览}
本笔记主要整理了李宏毅老师的机器学习课程、强化学习课程,以及部分大模型相关的学习内容。  

\subsection{学习目标}
\begin{itemize}
    \item 系统化整理课程知识点
    \item 配合课程PPT + 笔记 + 作业实现
    \item 便于后期复习和分享
\end{itemize}

% ---------------------------
\section{机器学习笔记}

\subsection{监督学习}
例如:线性回归、逻辑回归、神经网络等。

\textbf{线性回归公式:}
\[
y = Xw + b
\]

\subsection{代码示例}
\begin{lstlisting}[language=Python, caption=线性回归示例]
import numpy as np
from sklearn.linear_model import LinearRegression

X = np.array([[1],[2],[3],[4]])
y = np.array([2,4,6,8])

model = LinearRegression().fit(X,y)
print(model.coef_, model.intercept_)
\end{lstlisting}

% ---------------------------
\section{强化学习笔记}

\subsection{Q-Learning}
Q 学习的核心更新公式:
\[
Q(s,a) \leftarrow Q(s,a) + \alpha \Big[ r + \gamma \max_{a'} Q(s',a') - Q(s,a) \Big]
\]

\subsection{作业示例}
\begin{itemize}
    \item 实现一个简化版 Q-learning 算法
    \item 在 FrozenLake 环境上测试
\end{itemize}

% ---------------------------
\section{大模型笔记}

\subsection{Transformer 基本结构}
包含 Self-Attention、Feed-Forward、Residual Connection 等。

\subsection{公式总结}
注意力机制公式:
\[
\text{Attention}(Q,K,V) = \text{softmax}\left(\frac{QK^T}{\sqrt{d_k}}\right)V
\]

% ---------------------------
\section{总结与展望}
本笔记持续更新,后续会加入更多实验代码和论文阅读笔记。

\end{document}
